\section{Código fuente} 
\label{sec:codigo-fuente}

Esto es un ejemplo de anexo. En este anexo se muestran algunos ejemplos de como plasmar código fuente.

\begin{lstlisting}[language=Matlab]
function [T]=Ejercicio20(f,c)

T = char('B'*ones(8,8));

for i=1:8
    for j=1:8
        if ( (i==f) || (j==c) || (i+j==f+c) || (i-j==f-c) )
            T(i,j)='*';
        elseif ( rem(i+j,2)~=0 )
            T(i,j)='N';
        end
    end
end

T(f,c)='R';
\end{lstlisting}

\noindent O podemos quitar los números de línea y/o la raya:

\begin{lstlisting}[language=Matlab,frame=none,numbers=none]
function [T]=Ejercicio20(f,c)

T = char('B'*ones(8,8));

for i=1:8
    for j=1:8
        if ( (i==f) || (j==c) || (i+j==f+c) || (i-j==f-c) )
            T(i,j)='*';
        elseif ( rem(i+j,2)~=0 )
            T(i,j)='N';
        end
    end
end

T(f,c)='R';
\end{lstlisting}

\noindent Otro ejemplo:

\begin{verbatim}
function [T]=Ejercicio20(f,c)

T = char('B'*ones(8,8));

for i=1:8
  for j=1:8
    if ( (i==f) || (j==c) || (i+j==f+c) || (i-j==f-c) )
      T(i,j)='*';
    elseif ( rem(i+j,2)~=0 )
      T(i,j)='N';
    end
  end
end

T(f,c)='R';
\end{verbatim}

\noindent O con coloreado.

\begin{minted}{matlab}
function [T]=Ejercicio20(f,c)

T = char('B'*ones(8,8));

for i=1:8
    for j=1:8
        if ( (i==f) || (j==c) || (i+j==f+c) || (i-j==f-c) )
            T(i,j)='*';
        elseif ( rem(i+j,2)~=0 )
            T(i,j)='N';
        end
    end
end

T(f,c)='R';
\end{minted}
