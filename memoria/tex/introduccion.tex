\chapter{Introducción} 
\label{ch:introduccion}

En este capítulo debes introducir el problema sin divagar, sin copiar de otros documentos y sin utilizar un lenguaje excesivamente técnico.  Tampoco utilices un lenguaje informal.  Este capítulo debería convencer al cliente de que el proyecto merece la pena.  Es decir, es un problema real y no está resuelto completamente.

Redacta la introducción al final del TFG, cuando tengas elaborado el capítulo de antecedentes, los resultados y su discusión.  De esta forma podrás evitar repetir argumentos que ya están en esos capítulos.  La introducción debe introducir también el contexto en el que se desarrolla el TFG.

Utiliza secciones y subsecciones para organizar el contenido del capítulo.  Utiliza preferentemente frases cortas.

Por cierto, el cliente es el que paga.  En el TFG el que paga es el tribunal y tu director o directores.  Así que a quien tienes que convencer es a ellos.  El tribunal no lo conocerás a priori.  Por eso la memoria debe estar escrita para que la entienda alguien que no es especialista en el campo de aplicación.  Pero eso no implica que se toleren la falta de rigor o la falta de argumentación técnica.  Solo implica que los argumentos específicos hay que explicarlos o citar la fuente que los explica.


\section{Organización de la memoria} 
\label{sec:organizacion-memoria}

La organización de este documento es conforme al anexo TFG-08b~\cite{tfg08b} de la Normativa Trabajo Fin de Grado de la Escuela de Ingeniería Industrial de Toledo, de la Universidad de Castilla-La Mancha, aprobada en Junta de Escuela el 23 de junio de 2016. Se descompone en los siguientes capítulos.

\begin{description}
    \item[Capítulo~\ref{ch:antecedentes}] Analiza los antecedentes y estado del arte en relación al tema del proyecto.
    \item[Capítulo~\ref{ch:objetivos}] Enumera y justifica los objetivos del proyecto y establece los límites intrínsecos y extrínsecos de ejecución del TFG.
    \item[Capítulo~\ref{ch:contribuciones}] Enumera las principales contribuciones y aportaciones personales del autor en este TFG.
    \item[Capítulo~\ref{ch:procedimiento}] Describe la metodología de desarrollo empleada durante la ejecución del TFG.
    \item[Capítulo~\ref{ch:resultados}] Describe en detalle las pruebas realizadas y los resultados obtenidos.
    \item[Capítulo~\ref{ch:discusion-resultados}] Discute los resultados en relación a los objetivos del proyecto.
    \item[Capítulo~\ref{ch:conclusiones}] Recopila las principales conclusiones del proyecto.
    \item[Anexos] Complementan la información del cuerpo del documento con información técnica útil para reproducir los resultados, pero innecesaria para comprender en su totalidad el TFG realizado.
\end{description}

\warning{Al finalizar el resto de los capítulos revisa esta descripción del documento para que coincida con lo que realmente contiene la memoria.  Por ejemplo, es frecuente fusionar varios capítulos en uno cuando son muy pequeños.  También es frecuente lo contrario, dividir un capítulo en varios cuando es muy extenso.}

\section{Repositorio de información}
\label{sec:repositorio}

Todo el material generado durante la ejecución de este proyecto está disponible en \thegitrepo{}.  El repositorio incluye el código \LaTeX{} del presente documento, el código fuente de los programas realizados o modificados, y todos los datos generados en la evaluación de resultados.