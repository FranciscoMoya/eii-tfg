% Cambia los datos de tu TFG en este archivo

\title{Plantilla \LaTeX{} para elaborar el TFG en la Escuela de Ingeniería Industrial y Aeroespacial}
\author{Francisco Moya Fernández}

% IE = Ingeniería Eléctrica
% IEIA = Ingeniería Electrónica Industrial y Automática
% IA = Ingeniería Aeroespacial
\grado{IEIA}

% Tu número de expediente puedes consultarlo en secretaría virtual
\expediente{123456}

% En caso de múltiples directores separa los nombres con \\
% Si solo hay un tutor no pongas \\
\tutor{%
    Francisco Moya Fernández\\
    Fernando Castillo García}


% A partir de aquí se trata de datos opcionales. Si algún dato no quieres que figure borra la línea o coméntala poniendo el signo % al principio

% Es muy conveniente proporcionar un medio de contacto con el autor.  El correo electrónico es probablemente el menos invasivo. No uses correos con nicknames extraños, es un documento profesional. En caso de duda usa el correo de la Universidad, pero recuerda que dejará de ser válido unas semanas después de que dejes de ser alumno.
\email{tucorreo@alu.uclm.es}

% No uses tu teléfono personal, será accesible para cualquier usuario de la biblioteca
\phone{925 268 800 x.3729}

% Una página web para el proyecto puede ser un requisito necesario en caso de que sea trabajo parcialmente financiado con un proyecto de I+D. Consulta a tu tutor
\homepage{https://github.com/FranciscoMoya/eii-tfg}

% Tener un repositorio GIT (http://github.com) permite llevar un control de versiones. Tu tutor puede considerarlo esencial, habla con él
\gitrepo{https://github.com/FranciscoMoya/eii-tfg}

% Pon una dirección si puede ser interesante para recibir correspondencia relacionada. No pongas tu dirección personal
\address{UCLM --- Escuela de Ingeniería Industrial y Aeroespacial\\
    Campus Universitario de la Real Fábrica de Armas}
\poblacion{Toledo}
\cpostal{45071}
